\begin{description}

\item[C++ Code Generator:] The C++ Code Generator can automatically generate
  C++ code from your specification. You need a separate
  license for the \guicmd{C++ Code Generator} in order to access the
  \guicmd{C++ Code Generator} from the \Toolbox.
  
  
\item[Debugger:] With the debugger you can explore the behaviour of
  your specification. The debugger can execute the specification and
  break at function and
  \ifthenelse{\boolean{VDMsl}}{operation}{method} applications. At
  any point in the execution you can explore the local or global state
  and local identifiers in your specification.


%\ifthenelse{\boolean{VDMsl}}{}{
%\item[Dependency tool:] This tool gives an overview of inheritance and
%association information for a given class.}

\item[Dynamic Semantics:] The dynamic semantics describes the meaning
  of a language. Thus, the dynamic semantics describes
  how the language behaves if it can be executed.


\item[Emacs:] Emacs is an ASCII editor.


\item[GUI:] Graphical User Interface.


%\ifthenelse{\boolean{VDMsl}}{}{
%\item[Inheritance Tool:] The inheritance tool draws the inheritance
%  tree, that is the inheritance structure of your specification.}

\item[Interpreter:] The interpreter can interpret a specification
  according to the dynamic semantics of the language. That is, it can
  execute a program/specification. 


\ifthenelse{\boolean{VDMsl}}{}{
\item[Java Code Generator:] The Java Code Generator can automatically generate
  Java code from your specification. You need a separate
  license for the \guicmd{Java Code Generator} in order to access the
  \guicmd{Java Code Generator} from the \Toolbox.}
 
 
\item[\LaTeX:] is a generic typesetting system.


\item[Pretty Printer:] The pretty printer processes a file
  and produces a pretty printed version of the \vdmslpp\ parts in the
  input \vdmslpp\ file. The output format depends on the input format.


\item[Project:] A project is a collection of ASCII file names that make up
  a specification.


\item[RTF:] This is an acronym for ``Rich Text Format'' which is one
of the formats which can be used with the Microsoft Word editor.


\item[Semantics:] describes the meaning of the language. 


\item[Specification:] A specification is a \vdmslpp\ model of a system
  written in one or more files using potentially different input formats.
  
  
\item[Static Semantics:] The static semantics describes the
  relationships between the symbols of the language which must be
  obeyed in order for a syntactically correct specification to be
  well-formed (i.e.\ to have a consistent meaning). A well-formed
  specification is also called a type-correct specification.  


\item[Syntax:] The syntax of a language describes how the symbol
  elements of the language (e.g.\ key words and identifiers) can be related.
  The syntax only describes how the symbols can be ordered in the
  language, not the meaning of the ordering.


\item[Syntax Checker:] A syntax checker verifies if the syntax of a
  specification is correct.


\item[Test Coverage Information:] Information about how many times
  each construct in the specification has been executed.
  
  
\item[Test Coverage File:] The test coverage file contains test
  coverage information.
  
  
\item[Type Checker:] The type checker checks the type
  correctness of a specification. A specification can be definitely or
  possibly type correct.


\item[VDM:] The Vienna Development Method.


\item[\vdmsl:] The formal specification language of the {\em Vienna
    Development Method}. \vdmsl\ is an ISO standard
  language~\cite{ISOVDM96}. 


\item[\vdmpp:] An object-oriented specification language
 that is an extension of ISO VDM-SL.


\item[Well-formedness:] A specification can be well-formed with
  respect to the syntax and the static semantics of a language.


\end{description}
