\begin{description}

  
\item[C++コード生成機能:] 仕様書からC++のコードを自動生成する。
  \Toolbox\ から\guicmd{C++コード生成機能} に
  アクセスするためには別ライセンスが必要である。
  

\item[デバッガ:] デバッガを使えば仕様書の振る舞いを調査することができる。
  デバッガは仕様書を実行しアプリケーションの関数や
  \ifthenelse{\boolean{VDMsl}}{操作}{メソッド} でブレークすることもできる。
  実行中いつでも、仕様書内のローカルまたはグローバルな状態、ローカル変数などを調査することができる。\\

%\ifthenelse{\boolean{VDMsl}}{}{
%\item[Dependency tool:] This tool gives an overview of inheritance and
%association information for a given class.}


\item[動的セマンティクス:] 動的セマンティクスは言語の意味を記述する。
  すなわち動的セマンティクスは実行された場合に言語がどう振舞うかを記述する。


\item[Emacs:] ASCIIエディタ。


\item[GUI:] グラフィカルユーザーインターフェース

%\ifthenelse{\boolean{VDMsl}}{}{
%\item[Inheritance Tool:] The inheritance tool draws the inheritance
%  tree, that is the inheritance structure of your specification.}


\item[インタープリタ:] インタープリタは言語の動的動作に従って仕様書を解釈する。
  いわばプログラム/仕様書を実行する。\\

 
\ifthenelse{\boolean{VDMsl}}{}{
\item[Javaコード生成機能:] 仕様書からJavaのコードを自動生成する。
  \Toolbox\ から \guicmd{Javaコード生成機能} 
  にアクセスするには、別ライセンスが必要である。
}
 

\item[\LaTeX:] 一般的な組版システム


\item[清書機能:] ファイルを処理して\vdmslpp\ の入力ファイルの
  \vdmslpp\ の箇所の清書版を生成する。
  出力フォーマットは入力フォーマットに依存する。


\item[プロジェクト:] 仕様書を構成するASCIIのファイル名の集合


\item[RTF:] 「Rich Text Format」の頭字語。
  Microsoft Wordで使用できるフォーマットのひとつ。


\item[セマンティクス:] 言語の意味を記述したもの

  
\item[仕様書:] (おそらく)異なる入力フォーマットを使って書かれた
  1つ以上のファイルからなるシステムの \vdmslpp\ モデル
  

\item[静的セマンティクス:] 構文的に正しい仕様書を適格にするため
  (矛盾のない意味を持たせるため)に従わなくては
  成らない言語の記号間の関係を記述したもの。適格な仕様書とは型的に正しい仕様書とも言える。


\item[構文:] 言語の構文は、言語の記号要素(キーワード、識別子など)が
  どのように関連しているかを記述したものである。
  構文は言語中で記号がどのように命令されるかを記述しており、命令の意味を記述するものではない。


\item[構文チェック機能:] 仕様書の構文が正しいかどうか確認する。

  
\item[テストカバレッジ情報:] 仕様書の構成物が各々何回実行されたについての情報
  
  
\item[テストカバレッジファイル:] テストカバレッジ情報を含むファイル。
  

\item[型チェック機能:] 仕様書の型が正しいかどうかチェックする。
  'def"タイプと'pos"タイプ2種類のチェックがある。


\item[VDM:] ウイーン開発手法


\item[\vdmsl:] {\em Vienna Development Method}.の形式仕様言語。
  ISO標準言語である~\cite{ISOVDM96}。\\


\item[\vdmpp:] オブジェクト指向仕様言語。ISO VDM-SLの拡張。\\


\item[Well-formedness:] 仕様書が言語の構文、静的セマンティクスに関して適格であるということ。

\end{description}
