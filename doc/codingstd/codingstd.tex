\documentclass[11pt,a4paper]{article}

\usepackage{ifad}
% \usepackage{a4}

\newcommand{\TBW}{TO BE WRITTEN}

\begin{document}


\docdef{Coding Standards in the Toolbox Development}
         {IFAD VDM Tool Group \\
          The Institute of Applied Computer Science}
         {\today}
         {IFAD-VDM-56}
         {Report}
         {Draft}
         {Internal}
         {}
         {\copyright IFAD}
         {\item[1.0] Initial version, added porting experiences from
         DEC Alpha.}
         {\mbox{}}

\pagenumbering{roman}
\tableofcontents

\pagebreak
\renewcommand{\thepage}{\arabic{page}}
\setcounter{page}{1}
\parskip6pt
\parindent0pt


\section{Introduction}

This document describes guidelines that must be followed when
developing C++, VDM-SL or VDM++ code(/specification). The objective of
the coding standard is to improve the quality of code/specifications
developed within the Toolbox group. 


The strategy of introducing a coding standard into the development
work of the Toolbox is to start with a standard which easily can be
introduced and then step by step improve the coding standard to be
more and more specific. 

\section{Common Guidelines}

\subsection{Line Length}

\subsection{Header Files}

\subsection{Header Classes/Modules}

\subsection{Header Functions/Methods/Operations}


\section{VDM-SL}
In this section the coding standard for specifications written in
VDM-SL is described.

\section{Porting issues}
\label{sec:port}

This section describes what to take in consideration in order to write
code compilable on all platforms. It should furthermore be possible to
track the origin of a specific requirement. This way it will be 
possible to update the porting limitations due to change in platform
and/or compiler. 

\subsection{DEC cxx compiler on DEC Unix}
\label{sec:dec}

\subsubsection{Conditional Expression}

In conditional expressions should the invocation of method in a
temporary metaiv object be avoided. The following will give a
\texttt{ML} error, \textit{Null pointer detected}, on the DEC:

\begin{verbatim}
Sequence elseStmt(b_stmt.IsEmpty() ?
                  Sequence().ImpAppend(return_stmt) : 
                  <...> );
\end{verbatim}

This can be re-written this way:
\begin{verbatim}
Sequence appRet (Sequence().ImpAppend(return_stmt));
Sequence elseStmt(b_stmt.IsEmpty() ?
                  appRet :
                  <...>);
\end{verbatim}

\subsubsection{Method invocation of casted metaiv objects}

Method invocation of a casted metaiv object needs extra parentheses
around the cast. The \texttt{cxx}-compiler cannot compile the
following: 
\begin{verbatim}
Record (a).GetField (1);
\end{verbatim}

Extra parentheses should be added like this:
\begin{verbatim}
(Record (a)).GetField (1);
\end{verbatim}

\textbf{Note:} This restriction applies specific to version
\texttt{5.5} of the \texttt{cxx} compiler.

\subsubsection{Template casts}

Template casts is not supported by the DEC compiler. The following
cannot compile:
\begin{verbatim}
static_cast <IntVal&>(m);
\end{verbatim}

Instead a macro has been defined:
\begin{quote}
\begin{verbatim}
#ifdef __alpha 
#define STATIC_CAST(c) (c) 
#else 
#define STATIC_CAST(c)
static_cast<c> 
#endif // __alpha
\end{verbatim}
\end{quote}
Used this way:
\begin{verbatim}
STATIC_CAST (IntVal&)(m);
\end{verbatim}


\subsubsection{Misc. DEC}

\begin{itemize}
\item Output string stream has no \texttt{freeze} method
\item \texttt{pair}-template should be initialised when
  declared. This:
\begin{verbatim}
   pair<MapVal::MapType::iterator, bool> p;
   p = value.insert( pair<const Generic, Generic> (dom_el, ran_el) );
\end{verbatim}
  should be changed to:
\begin{verbatim}
  pair<MapVal::MapType::iterator, bool> 
  p( value.insert( pair<const Generic, Generic> (gdom, gran) ));
\end{verbatim}
\item In \texttt{exc\_emul.h}, the type of \texttt{jmp\_buf\_type} is
  \texttt{long} on the \textit{Alpha}-architecture (\texttt{int} on
  all other platforms).
\item Type \texttt{bool} is not directly supported by the compiler,
  instead it is defined by including \textit{STL}-header
  \texttt{vector}. 
\item On DEC there is no global \textit{STL} header file, instead each
  specific template header must be included as needed (file
  \texttt{metaiv.cc}): 
\begin{verbatim}
  #ifndef __alpha
  #include <stl.h>
  #else
  #include <map>
  #include <vector>
  #include <set>
  #endif // __alpha
\end{verbatim}

\end{itemize}
\end{document}





