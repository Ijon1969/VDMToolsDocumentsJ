\documentclass[]{article}
\usepackage{isolatin1, a4wide,epsfig}

\setlength{\parskip}{\baselineskip}
\setlength{\parindent}{0cm}
\begin{document}

\section{Recursive Debug in the Interpreter. - IFAD-VDM-61}
\label{sec:first}
In the VDM-SL interpreter and in the non parallel VDM-PP interpreter, it is
today possible to invoke a debug command or a print command
recursively. This is very useful for inspecting variables, and for
``trying'' the next command, before it is actually executed by the
interpreter.

Three issues need to be addressed when the above functionality is offered
from the toolbox, namely:
\begin{enumerate}
\item Should the changes made to the state by the recursive debug call
``survive'' after the debug call has finished and execution returns to the
original breakpoint.
\item What should happen when a runtime error occur in the recursive debug
call
\item How should the recursive debug levels be represented in the GUI.
\end{enumerate}

Today these three questions are answered with:
\begin{enumerate}
\item Yes, changes should survive (that is, this is the case for our
  toolboxes today).
\item When a runtime error occurs, then it should be possible for the user
  to continue his original debugging. This is essential as inspection of
  variables also are made using recursive debug (and misspelling a variable
  name should not force the user to restart debugging all over again.)
  Furthermore, it would be very useful if it was possible to inspect
  variable in the scope of the runtime error. These two wishes together
  forms the need for a command called \texttt{popd} (that is, pop-debug).
  Optional this command could be executed implicit by the toolboxes when
  the user invokes \texttt{step}, \texttt{continue} or one of the other
  debugger commands.
\item A Suggestion to this issue, is that each level of recursive debug is
  shown on the call stack just like ordinary recursive function calls. This
  can be seen in figure 1.
\end{enumerate}

\textbf{NOTE}: The above is part how it is today, and part wishes for how
it could be, and all of it only relates to the non-parallel part of the interpreter.




\section{Conceptually view of recursive debugging.}
\label{sec:concep_view}

Whenever a recursive debug command is issued, then this may conceptully be
viewed as if the debug command had been part of the
original program as the next line. To see it that way, a few items needs
to be pointed out:

\begin{itemize}
\item Program execution must stop after the recursive debug command, and
  the result of its evaluation must be printed.
\item This view makes sure that the scope is OK, namely that the scope is
  the current function. Therefore it is possible to type ``print var''.
\item In contrast to if the expression had been part of the original
  program literally, a runtime error meet during recursive debugging will
  not make the whole program stop.
\end{itemize}




\section{Should threads started in recursive debugging survive?}
\label{sec:survive}

When introducing the recursive debugging into the parallel part, a few
additional issues need to be addressed. First of all the command issued may
start new threads. Should these threads be kill when the recursive debug
command returns? 

The following is an example, which may be used to discuss this issue:

\begin{verbatim}
class Counter

instance variables
  counter : nat

thread
  while true do
    counter := counter + 1    

end Counter



class A

instance variables
  myref : Counter

operations

start: () ==> ()
start() ==
  (myref := new Counter();
   start(myref));

restart: () ==> nat
restart() ==
  let oldcount = obj.counter,
  in ( myref := new Counter()
      start(myref);
      return oldcount)

end A
\end{verbatim}

In this example, the class \texttt{Counter} is a simple counter which does
nothing else than define a thread, which counts its instance variable up
in an infinity. 

The class A uses this counter and it has two operations, which may start
and restart the counting respectively. \footnote{In a real implementation
these two operations should have been located in the counter class, please
forgive me for not finding a good example}
Starting and restarting the counter is done by creating a new
\texttt{Counter} object, and starting its thread.

The following is a debugging session, where the second debug command is
invoked recursively. (Though this fact should not change much on the discussion).

\begin{verbatim}
vdmpp> cr obj := new A()
vdmpp> debug obj.start()
vdmpp> debug do-some-command
vdmpp> debug obj.restart();   <-- issued recursively
\end{verbatim}

If the threads started from any of the debug commands are terminated when
the command is finished, then the system above would not work, as the
counter thread would be terminated when the \texttt{restart} command was
finished executing its instructions.

The conclusion must therefore be: 
\begin{quote}
Threads started by recursive debugging commands should not be terminated
when the recursive debug command terminates. 
\end{quote}

The rules for normal termination of threads do of course still apply. That
is, when an object with an active thread goes out of scope, then the thread
will be terminated.





\section{Which Processes to Schedule when doing Recursive Debugging.}
When invoking a recursive debugging command the scheduler should continue scheduling each of the threads, the question is, however, which threads should be schedules?

The following reasonable suggestions exists:
\begin{enumerate}
\item Only the new thread should be scheduled, all other threads should
simply be ignored, and should first be rescheduled whenever the recursive
debugging is finished.
\item All threads should be scheduled including the thread with the
breakpoint (that is the thread where a breakpoint was meet, allowing a
recursive debugging to be started). The recursive debugging should be seen
as a new thread.
\item All threads excluding the thread with the breakpoint should be scheduled.
\end{enumerate}

The function which may be invoked from the recursive debug command may
require that the other threads are running. An example could be that it
meets a history guard, which will not allow it to continue before some
other thread has left a given operation. Therefore other threads needs to
be running.

The question is now only, should the thread which has meet a breakpoint be
scheduled or not?

>From the conceptual view in section \ref{sec:concep_view}, the recursive
debug command should be seen as \emph{the next line of code} in the given
thread. This implies that the thread which have meet a breakpoint should
\textbf{not} be scheduled.

Fortunately, this also implies that we have a well defined location to
break when the recursive debug command has finished, just as it is the case
today in the existing tools. 


\section{Runtime Errors meet in the Initial Debugging Command.}
What should happen if a runtime error occurs in any of the threads which
the original debug command is parent to?

An answer to this question could be that the whole debug session should
terminate. This answer is based on the view that all the threads combine
one whole system, and a runtime error in one part of the system means that
the whole system has entered an undetermined state.

An example where this view seems reasonable is a system where a timer is
executing in its own thread. If this timer meets a runtime error and the
tread is terminated, then a vital part of the system has stopped, and the
concept of time is missing for the other component of the system.

Another possible view of runtime errors in threads may be that it is up to
the user to verify if the rest of the system may continue. If he wishes to
continue debugging then he just ask the system to continue. The thread with
the runtime error will then be discarded and the system will continue
scheduling the rest of the threads.

A silly example of this could be a simulation of a consumer/producer system
with a number of consumers, and a timer attached to the system. This timer
will be used to probe the idle time of each of the consumers. The system
may model \emph{to much idle time - fire one of the consumers} by letting a
consumer thread throw a runtime error if the given consumer has been idle
for too long.

In this system the user may want to continue after a runtime error with one
less consumer.

It is obvious that continuing after a runtime error may lead to a state of
the system where \emph{strange things may happen}! And using runtime errors
in the above way seems to be a dirty hack, which could be avoided. 

Therefore I find that a runtime error should mean ``\emph{the whole program has
meet a runtime error}'' rather than just ``\emph{the given thread has meet a
runtime error}''.


\section{Runtime Errors meet in recursive Debugging Commands.}
Having discussed runtime errors which stems from the initial debug command,
the question is now: \emph{what should happen if a runtime error
occur in a recursive debug command?}

An obvious answer could be that all the threads started from the recursive
debug command should be terminated when a runtime error occur in any of
these threads - This is just like the answer from above. 

This is, however, no good as the new thread started by the recursive debug
command may now be part of the original system, as was the case in section
\ref{sec:survive}, and if they are terminated then the whole system will be
effected.

Therefore in general we can not decide which threads, started from a
recursive debug command, may be killed when a runtime error occurs. 

A solution to this problem is to view all the sub-threads to the recursive
debug thread as being part of the \textbf{initial} system rather than being part of
the recursive debug command. Thus when a runtime error occur in any of
these threads then the whole system will terminate. If on the other hand a
runtime error occur in the recursive debug thread then only this thread will
terminate.

This solution requires that we can distinguish between a runtime error in
the latest debug command and in the original debug command.

What is even worse, the user needs to understand the difference between a
runtime error in the original debug threads and in its sub-threads. In
other words, sometimes he may recover from a runtime error in recursive
debug commands and sometimes he may not (He may not recover when the error
occurs in any of the sub-threads as they are now regarded as being part of
the initial system).

A less ambitious solution to this problem could be to forbid new threads to
be started from a recursive debugging command. If the user invokes a
recursive debug command, which starts a new thread, then a runtime error
may simply be thrown, to signal that this is not allowed.

Again we have a non-symmetric solution, but this time the rules are much
easier to understand from the users point of view, namely: ``\emph{in
recursive debug commands you are not allowed to start new threads.}''

\section{Representing recursive debugging in the GUI.}
Finally we need to discuss how recursive debugging for the parallel part of
VDM-PP may be represented in the GUI. 

We may reuse the strategy from the original approach discussed in section
\ref{sec:first}, as can be seen in figure 1, with the change that each
thread have their separate debug stack. This can be seen in figure 2.

\textbf{NOTE}: In figure 2 a recursive debug command is done \emph{in the
thread} rather than starting a new thread as has been done in the previous
discussion.

Typing \texttt{popd} can only be done when the thread in scope contains a
recursive debug command, and this command then means that the given threads
recursive debug command should be poped.

As threads started in recursive debugging commands are part of the initial
system, it is not possible to invoke \texttt{popd} in these
threads. Therefore allowing creation of new threads in recursive debug
commands, do not change the behavior of how it should be shown in the GUI,
as long as the basic assumption described in section \ref{sec:concep_view}
is kept. Namely that starting a recursive debugging command is just like
inserting the command as the next line of the original program. Which
implies that it is not possible to have the original thread and the
recursive debugging command running at the same time.


\end{document}

